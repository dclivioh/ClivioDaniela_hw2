%--------------------------------------------------------------------
%--------------------------------------------------------------------
% Formato para los talleres del curso de Métodos Computacionales
% Universidad de los Andes
%--------------------------------------------------------------------
%--------------------------------------------------------------------

\documentclass[11pt,letterpaper]{exam}
\usepackage[utf8]{inputenc}
\usepackage[spanish]{babel}
\usepackage{graphicx}
\usepackage{tabularx}
\usepackage[absolute]{textpos} % Para poner una imagen en posiciones arbitrarias
\usepackage{multirow}
\usepackage{float}
\usepackage{hyperref}
%\decimalpoint

\begin{document}
\begin{center}
{\Large Métodos Computacionales} \\
\textsc{Tarea 2}\\
01-2019\\
Daniela Clivio Hernández\\
201217858\\
\end{center}

\noindent
\section{Ejercicio 1: Transformada de Fourier: imprementación propia, paquetes de scipy y espectrogramas.}
En esta gráfica observamos la señales dadas contiguas y superpuestas, donde se observan dos frecuencias dadas.\
\begin{figure}[H]
\centering
\includegraphics[height=15cm]{grafica_senales.pdf}
\caption{Señales signal.dat y signalSuma.dat}
\label{fig1}
\end{figure}

Se aplica la transformada de Fourier a las señales dadas y obtenemos las amplitudes en función de las posiciones de las frecuencias. Acá observamos donde se presentan las dos frecuencias, sin embargo no conocemos los valores de estas frecuencias. Se utiliza una implementación propia para recuperar los valores de las frecuencias, por medio de la ecuacion $k/n*dt$ donde k es un arreglo del tamaño de la señal, n es el tamaño de la señal y dt el paso del tiempo.\\
\begin{figure}[H]
\centering
\includegraphics[height=16cm]{grafica_transformadas.pdf}
\caption{Transformada de Fourier aplicada a las señales.}
\label{fig2}
\end{figure}

Se grafican los espectrogramas de las señales. Este gráfico 3D, nos muestra en el eje x el tiempo, en el eje y las frecuencias y en el color, la tercera dimensión, las amplitudes. Se observan los picos de amplitud entre 0 y 500, lo cual se ve representado en los espectrogramas con las lineas rectas. Así mismo en el primer espectrograma se observan cada una de las señales independientemente así como el cambio entre ellas. Mientras que en el segundo espectrograma se ve una gráfica más homogénea que se muestra casi que como una sola señal. Sin embargo se mantienen los picos entre 0 y 500. \\
\begin{figure}[H]
\centering
\includegraphics[height=16cm]{grafica_espectrogramas.pdf}
\caption{Espectrogramas de las frecuencias en función del tiempo.}
\label{fig3}
\end{figure}

Observamos la señal generada por datos reales tomados de un temblor con un cambio brusco alrededor de 500 en el tiempo. \\
\begin{figure}[H]
\centering
\includegraphics[height=9cm]{grafica_senaltemblor.pdf}
\caption{Señal de un temblor real.}
\label{fig4}
\end{figure}

Se grafica la transformada de Fourier aplicada a la señal del temblor.\\
\begin{figure}[H]
\centering
\includegraphics[height=9cm]{grafica_transformadatemblor.pdf}
\caption{Transformada de la señal del temblor.}
\label{fig5}
\end{figure}

Obtenemos el espectrograma de la señal donde se tiene lo que parece ser un salto entre 500 y 600, lo que equivale a la disminución de la señal que se observa en la figura \ref{fig4} en la transición de las dos amplitudes fuertes predominantes. \\
\begin{figure}[H]
\centering
\includegraphics[height=9cm]{grafica_espectrogramatemblor.pdf}
\caption{Espectrograma del temblor.}
\label{fig6}
\end{figure}

\noindent
\section{Ejercicio 2: Ecuaciones diferenciales ordinarias: un edificio en un sismo.}

En estas gráficas tenemos el comportamiento de la amplitud en función del tiempo para determinados valores de w. Para el primer valor de w, w1=0.6, observamos una señal con gran periodo y amplitud que corresponde al primer modo. Para w2=1.8 estamos en el pico del segundo modo, donde vemos una señal con un periodo menor y más constante. Para w3=2.6 tenemos una señal similar al w2 pero con mayor frecuencia que corresponde al pico del tercer modo. Por último, para w4=4.0 no encontramos un modo en el que predomine la vibración de las masas  según las posibilidades físicas.\\
\begin{figure}[H]
\centering
\includegraphics[height=15cm]{grafica_ui.pdf}
\caption{Amplitudes máximas en función del tiempo para cada piso.}
\label{fig7}
\end{figure}

En la gráfica de las amplitudes máximas en función del tiempo, observamos tres picos. Estos picos equivalen a tres frecuencias asociadas a tres modos normales que corresponden a las formas posibles de vibrar de las masas. En el piso 1 observamos que hay un mayor aporte por parte de la frecuencia del segundo modo normal, el piso dos tiene mayor tendencia a los modos 1 y 3. Mientras que el piso 3 presenta predominancia de los dos primeros modos. \\
\begin{figure}[H]
\centering
\includegraphics[height=15cm]{grafica_wAmax.pdf}
\caption{Amplitudes máximas en función de las frecuencias}
\label{fig8}
\end{figure}

\end{document}
