%--------------------------------------------------------------------
%--------------------------------------------------------------------
% Formato para los talleres del curso de Métodos Computacionales
% Universidad de los Andes
%--------------------------------------------------------------------
%--------------------------------------------------------------------

\documentclass[11pt,letterpaper]{exam}
\usepackage[utf8]{inputenc}
\usepackage[spanish]{babel}
\usepackage{graphicx}
\usepackage{tabularx}
\usepackage[absolute]{textpos} % Para poner una imagen en posiciones arbitrarias
\usepackage{multirow}
\usepackage{float}
\usepackage{hyperref}
%\decimalpoint

\begin{document}
\begin{center}
{\Large Métodos Computacionales} \\
\textsc{Tarea 2}\\
01-2019\\
Daniela Clivio Hernández\\
201217858\\
\end{center}

\noindent
\section{Ejercicio 1: Fourier}
\begin{figure}[H]
\centering
\includegraphics[height=16cm]{grafica_senales.pdf}
\caption{Señales signal.dat y signalSuma.dat}
\label{fig1}
\end{figure}

\begin{figure}[H]
\centering
\includegraphics[height=16cm]{grafica_transformadas.pdf}
\caption{Transformada de Fourier aplicada a las señales.}
\label{fig2}
\end{figure}

\begin{figure}[H]
\centering
\includegraphics[height=16cm]{grafica_espectrogramas.pdf}
\caption{Espectrogramas de las frecuencias en función del tiempo.}
\label{fig3}
\end{figure}

\begin{figure}[H]
\centering
\includegraphics[height=9cm]{grafica_senaltemblor.pdf}
\caption{Señal de un temblor real.}
\label{fig4}
\end{figure}

\begin{figure}[H]
\centering
\includegraphics[height=9cm]{grafica_transformadatemblor.pdf}
\caption{Transformada de la señal del temblor.}
\label{fig5}
\end{figure}

\begin{figure}[H]
\centering
\includegraphics[height=9cm]{grafica_espectrogramatemblor.pdf}
\caption{Espectrograma del temblor.}
\label{fig6}
\end{figure}

%Analisis de resultados...

\end{document}
